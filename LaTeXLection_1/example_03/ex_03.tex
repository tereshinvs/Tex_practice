%
%
% 	Пример 03
%

%
%	С этой команды начинается каждый документ LaTeX: она задаёт общие настройки формата документа
%	Формат бумаги А4, 12й кегль (размер шрифта)
%
\documentclass[a4paper,12pt]{article}

%
%	Эти две команды подключают локализацию
%
\usepackage[utf8]{inputenc}			% Кодировка символов --- Юникод.
\usepackage[english, russian]{babel}		% Язык: шрифты, переносы и т.д.

%
%	Эти команды регулируют внешний вид документа
%

\usepackage{indentfirst}	% Абзацный отступ после начала глав и разделов
\usepackage{a4wide}		% Сужает поля


\begin{document}
%
%	Пример из книги Братусь, Новожилов, Платонов, ``Динамические системы и модели биологии''
%
%	Пустая строка разделяет отдельные абзацы.
%	Использованы символы:
%	-		дефис	(часть слова, ``во-первых'', ``как-нибудь'')
%	--		тире-без-отбивки (сдвоенные слова: ``красно--синий'', ``неравенство Коши--Буняковского'' и перечисления, ``в 2007--2012 годах'')
%	---		тире 	(замена ``это'', ``есть'': ``Фортран --- язык программирования'')
%

\tableofcontents

\section{sdfsdg}

\section{Образец текста, начало}
При  написании этой книги авторы руководствовались двумя взаимосвязанными
задачами. Первая --- показать разнообразие математических моделей, описывающих
биологические сообщества. Вторая --- описать основные математические методы качественного анализа нелинейных систем. По мере возможности, авторы стремились
сочетать строгое изложение математической теории с конкретными приложениями.
Математические модели, рассмотренные в книге, можно грубо классифицировать
на конечномерные с дискретным временем (разностные уравнения), конечномерные
с непрерывным временем (системы обыкновенных дифференциальных уравнений)
и бесконечномерные (уравнения в частных производных и интегро-дифференциальные уравнения). Все эти классы, отражающие последовательные стадии точного
отображения биологической реальности. При этом, за рамками исследований остались модели, описываемые с помощью методов теории вероятностей и методов имитационного моделирования.

Первая часть книги, по существу, представляет учебник по курсу математических
моделей биологии, где в конце каждой главы содержание подкрепляется упражнениями, часть из которых представляет самостоятельный научный интерес.
Здесь рассматриваются классические математические модели, вошедшие в золотой фонд математической биологии (например, модель Лотки--Вальтерры, Гаузе,
модель распространения эпидемий Кермака--Маккендрика и т.д.). С другой стороны, в книге содержится большое число моделей, которые описаны лишь в специальных журнальных публикациях и модели, которые были предложены совсем недавно
(например, модель эволюции генных семейств, модель распространения эпидемий в
неоднородных популяциях, модель распределенного гиперцикла).

\section{Образец текста, продолжение}
Как правило, биологические системы описываются нелинейными соотношениями и содержат параметры, значения которых либо неизвестно, либо их определение
сопряжено со значительными трудностями. Особенно важным оказывается исследование поведения системы вблизи тех значений параметров, при которых возникают
перестройки (возможно и катастрофичекие) в поведении биологических сообществ.
Поэтому часть материала книги посвящена изложению основ теории бифуркаций.
Качественные методы анализа нелинейных динамических систем излагаются достаточно подробно и иллюстрируются многочисленными примерами, однако доказательства многих утверждений и теорем опущены. Для полноценного изучения математических методов, изложенных в книге, мы отсылаем читателя к специальной
литературе. Для удобства читателей, ряд утверждений и сложные математические
выкладки вынесены в приложения.

Процесс построения математических моделей не поддается полной алгоритмизации; в
большей степени, это ремесло, которое сродни искусству. При этом, важной проблемой остается мера адекватности модели изучаемому явлению. Здесь исследователь находится между Сциллой точного отображения природы путем ее детального
описания и Харибдой невозможности исследования математической модели в силу
ее трудности. Мы надеемся, что предлагаемая книга может помочь в постижении
приемов и методов математического моделирования биологических сообществ.


\end{document} 

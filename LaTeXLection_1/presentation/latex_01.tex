\documentclass[ucs, mathserif]{beamer}
\usepackage[utf8x]{inputenc}
\usepackage[english, russian]{babel}
\usepackage{indentfirst}

\RequirePackage{paralist}

\usepackage{amsmath}
\usepackage{amsthm}
\usepackage{amssymb}
\usepackage{amscd}
\usepackage{bm}

\newcommand*{\HM}[1]{#1\nobreak\discretionary{}{\hbox{$\mathsurround=0pt #1$}}{}}

\usetheme[alternativetitlepage=true, titleline=true, bullet=circle]{Torino}
\logo{\includegraphics[height=30px]{msu}}

\usepackage{ragged2e}
\justifying
\renewcommand{\raggedright}{\leftskip=0pt \rightskip=0pt plus 0cm}

\title[Первая лекция по \LaTeX]{Основы работы в \LaTeX}
\author[Кафедра СА]{Кафедра системного анализа}
\institute{МГУ им. М.В. Ломоносова \\ Факультет вычислительной математики и кибернетики}
\date{Сентябрь 2013}

\begin{document}
\maketitle
%
%
%
\begin{frame}{Что такое \LaTeX? (Взгляд снаружи)}
\begin{itemize}
 \item \LaTeX\ --- система подготовки (вёрстки) текстов на базе \TeX.
 \item \LaTeX\ <<заточен>> для простой работы с формулами и ссылками.
 \item \LaTeX\ реализован на всех платформах и распространяется абсолютно бесплатно.
 \item \LaTeX\ является де-факто международным стандартом для подготовки и представления естественнонаучных текстов.
\end{itemize}
 
\end{frame}
%
%
%
\begin{frame}{Что такое \LaTeX? (Взгляд изнутри)}
\begin{itemize}
 \item \LaTeX\ --- язык описания и разметки содержимого документа. 
 \item \LaTeX\ --- не WYSIWYG. Разделяются файлы \emph{исходного кода} (создаваемого пользователем) и файлы результирующего \emph{документа} (создаваемые из исходников транслятором).  
 \item Возможности \LaTeX\ существенно зависят от \emph{библиотеки пакетов}, предоставляемой конкретным \emph{дистрибутивом}.
 \item Процесс транслирования исходного кода в документ осуществляется в один проход. Подавляющее большинство команд языка --- \emph{макросы}.
\end{itemize}
 
\end{frame}
%
%
%
\begin{frame}{Литература}
\begin{thebibliography}{99}
 \bibitem{newllang} Львовский~С.~М. \emph{Набор и верстка в системе \LaTeX.} М.:~МЦНМО, 2007.
 \bibitem{stol} Столяров~А~В. \emph{Сверстай диплом красиво: \LaTeX\ за три дня.} М.:~Макс Пресс, 2010.
 \bibitem{GM} Гуссенс M., Миттельбах  Ф., Самарин А. \emph{Путеводитель по пакету \LaTeX\ и его расширениям.} М.: Мир, 1999.
\end{thebibliography}
\end{frame}

%
%
%
\begin{frame}{Дистрибутивы}

Дистрибутив предоставляет набор трансляторов (программы \texttt{latex}, \texttt{pdflatex}), конвертеров (\texttt{dvips}, \texttt{ps2pdf}), библиотеку шрифтов и пакетов, а также другие полезные утилиты (просмотрщики, программы для работы с базами ссылок и т.д.)

Основные дистрибутивы:
\begin{description}
 \item[UNIX.] TeXLive (входит в репозитории большинства ОС, в т.ч. Fedora, Debian, Ubuntu, Gentoo, FreeBSD, Mac OS)
 \item[Windows.] MiKTeX (\url{http://miktex.org/})
\end{description}
 
\end{frame}

%
%
%
\begin{frame}{Процесс работы}
Процесс построения документа в \emph{векторных} форматах:
 \begin{itemize}
  \item \texttt{test.tex}: cоздаётся текстовым редактором.
  \item \texttt{test.dvi}: команда \texttt{latex test.tex}
  \item \texttt{test.ps}: команда \texttt{dvips test.dvi}
  \item \texttt{test.pdf}: команда \texttt{ps2pdf test.pdf}
 \end{itemize}

Для работы с \emph{растровой} графикой: \texttt{pdflatex test.tex} 

В одном документе растровую графику с векторной совмещать нельзя.
\end{frame}

%
%
%
\begin{frame}{Графические среды (IDE)}
 \begin{itemize}
 \item Kile (KDE, xfce; свободный)
 \item Texmaker (*NIX, Windows, Mac OS; свободный)
 \item WinEdt (Windows; проприетарный)
 \item TeXShop (Mac OS; свободный)
 \item Eclipse с плагином \TeX (*NIX, Windows, Mac OS; свободный)
\end{itemize}

Подробная сравнительная таблица: 

\url{http://en.wikipedia.org/wiki/Comparison_of_TeX_editors}

MiKTeX + WinEdt без необходимости настройки для Windows:

\url{http://rutracker.org/forum/viewtopic.php?t=3865802}

\end{frame}

%
%
%
\begin{frame}[fragile]{Структура документа}
\begin{verbatim}
\documentclass[options]{classtype}
%
%	Преамбула:
%	Подключение пакетов, опций, настроек,...
%
\begin{document}
%
%	Код, описывающий содержимое документа
%
\end{document}
\end{verbatim}

\end{frame}

%
%
%
\begin{frame}[fragile]{Русификация}
Для подключения русского языка надо добавить в преамбулу строчки:
\begin{verbatim}
\usepackage[Кодировка]{inputenc} 
\usepackage[english, russian]{babel}
\end{verbatim}
где <<кодировка>> может принимать значения:
\begin{itemize}
	\item \texttt{utf8} --- универсальная многоязыковая кодировка. Кодировка текстовых файлов по умолчанию для многих ОС, в т.ч. --- Debian, Ubuntu, Mac OS. Рекомендована к использованию.
	\item \texttt{cp1251} --- стандартная русскоязычная кодировка Windows. За её пределами не встречается.
	\item \texttt{koi8-r} --- архаичная кодировка для *NIX систем. Сейчас практически не используется.
\end{itemize}
\end{frame}
%
%
%
\begin{frame}[fragile]{Команды и окружения}

Все команды начинаются с символа \verb:\:

Имена команд могут состоять \emph{только} из латинских букв.

Служебные символы: \verb:{ } $ @ & # % _:

Команды с аргументом: \verb:\commandName{Arg1}{Arg2}{Arg3}:

Команды с необязательным аргументом: 

\verb:\commandName[OptArg1]{Arg1}:

\textit{Окружение}: набор команд, заключенных между фигурными скобками (безымянное окружение) или конструкция вида

\begin{verbatim}
\begin{envName}
%
%	Команды
%
\end{envName}
\end{verbatim}
\end{frame}

\begin{frame}[fragile]{Набор формул}
 Строчные формулы оформляются с помощью символов долларов в начале и конце: например, код
\verb:$\dot{x}(t) = A(t)x(t)$: породит формулу $\dot{x}(t) = A(t)x(t)$. Выносные \emph{ненумерованные} формулы получаются с помощью пар сдвоенных долларов:
\begin{verbatim}
$$
\sum\limits_{k=0}^{n}\dfrac{1}{k} = c + \ln (n) + o(n)
$$
\end{verbatim}
получаем
$$
\sum\limits_{k=0}^{n}\dfrac{1}{k} = c + \ln (n) + o(n)
$$
\end{frame}

\begin{frame}[fragile]{Ссылочный механизм}
 Нумерованные выносные формулы можно получать с помощью окружений \texttt{equation}, \texttt{gather}, \texttt{multline} и других. Каждое окружение предназначено для своего типа формул.

Общая идея ссылочного механизма: у объекта, содержащий номер (формулы, раздела, рисунка, таблицы, \ldots) ставится команда \verb:\label{text}:, после чего за объектом закрепляется метка \verb:text:. Сослаться на объект можно многими способами: командами \texttt{ref} (просто номер), \texttt{pageref} (ссылка на страницу документы, где помещён объект) и \texttt{eqref} (для формул).

\end{frame}

%
%
%
\begin{frame}{Домашнее задание}
 \begin{itemize}
	\item  Читать \cite{newllang}, стр. 11--33, 45--89.
	\item  Установить \LaTeX.
	\item  Составить документ, в котором содержалась бы следующая информация:
	\begin{itemize}
		\item Имя и фамилия автора документа;
		\item Название дистрибутива \LaTeX, среды разработки и операционной системы, использованных для создания документа;
		\item Пять любых теорем (лемм), обязательно содержащие формулы, по одной из курсов математического анализа, линейной алгебры, дифференциальных уравнений, теории вероятностей и дискретной математики.
	\end{itemize}
	и прислать его на почту \url{month_october@mail.ru} для проверки. Задание должно быть выполнено в соответствии с Требованиями по оформлению отчётов в \LaTeX.

 \end{itemize}
\end{frame}

\end{document}

%
%
% 	Пример 04
%

%
%	С этой команды начинается каждый документ LaTeX: она задаёт общие настройки формата документа
%	Формат бумаги А4, 12й кегль (размер шрифта)
%
\documentclass[a4paper,12pt]{article}

%
%	Эти две команды подключают локализацию
%
\usepackage[utf8]{inputenc}			% Кодировка символов --- Юникод.
\usepackage[english, russian]{babel}		% Язык: шрифты, переносы и т.д.

%
%	Эти команды регулируют внешний вид документа
%

\usepackage{indentfirst}	% Абзацный отступ после начала глав и разделов
\usepackage{a4wide}		% Сужает поля

%
%	Эти команды подключают много полезного для работы с формулами
%

\usepackage{amsmath}
\usepackage{amsthm}
\usepackage{amssymb}

%
%	Основное тело документа
%
\begin{document}

\section{Текстовые окружения и команды}\label{textInfo}

\textit{Курсивное}, \textbf{жирное} и \texttt{моноширинное} начертание. Вариант 1.

{\it Курсивное}, {\bf жирное} и {\tt моноширинное} начертание. Вариант 2.

\begin{center}
 Это окружение выравнивает текст, содержащийся в нём, по центру страницы. Это окружение выравнивает текст, содержащийся в нём, по центру страницы. Это окружение выравнивает текст, содержащийся в нём, по центру страницы.
\end{center}


\section{Формулы}

Вот это предложение содержит одно известное уравнение в виде \textit{строчной} формулы: $a^n+b^n\neq c^n$ для всех $a,\,b,\ c,\; n\in \mathbb{N}$.

$\int_a^b$ А вот это --- пример выносной ненумерованной формулы:

$$
\int\limits_a^bf(x)\,dx = \lim\limits_{\delta\to0}\sum\limits_{k=1}^{N}f(x_k)(x_k - x_{k-1}).
$$

\begin{equation}
asdasfas	
\end{equation}

Далее --- пример использования нумерованной формулы:
\begin{equation}\label{lawOfLargeNumbers}
 \left(\dfrac{\sum\limits_a^b\xi_1 + \xi_2 + \dots + \xi_N}{N}\right) \xrightarrow[N\to\infty]{} \mathbb{E}\xi_1,
\end{equation}
на которую мы можем сослаться, воспользовавшись командой \verb:\eqref:: 

Формула \eqref{lawOfLargeNumbers} представляет собой закон больших чисел.


Формула asdsafasfsafasfasfasfasfasfasfasfagagfadg sdfsdgdsgsdgsdgsdgsgsdgadsfsdggs(\ref{lawOfLargeNumbers}) представляет собой закон больших чисел.

\fbox{Общее правило --- если на формулу нет ссылки, не надо нумеровать её!}

Для ссылки на не-формулы надо пользоваться командой \verb:\ref::

В разделе \ref{textInfo} были описаны некоторые команды для работы с текстом в \LaTeX.



\end{document} 

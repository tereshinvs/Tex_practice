\documentclass[ mathserif]{beamer}
\usepackage[utf8x]{inputenc}
\usepackage[english, russian]{babel}
\usepackage{indentfirst}

\RequirePackage{paralist}

\usepackage{amsmath}
\usepackage{amsthm}
\usepackage{amssymb}
\usepackage{amscd}
\usepackage{bm}

\newcommand*{\HM}[1]{#1\nobreak\discretionary{}{\hbox{$\mathsurround=0pt #1$}}{}}

\usetheme[alternativetitlepage=true, titleline=true, bullet=circle]{Torino}
\logo{\includegraphics[height=30px]{msu}}

\usepackage{ragged2e}
\justifying
\renewcommand{\raggedright}{\leftskip=0pt \rightskip=0pt plus 0cm}

\title[Вторая лекция по \LaTeX]{Продолжение основ работы в \LaTeX}
\author[Кафедра СА]{Кафедра системного анализа}
\institute{МГУ им. М.В. Ломоносова \\ Факультет вычислительной математики и кибернетики}
\date{Сентябрь 2013}

%
%
%
\newcommand{\R}{\ensuremath{\mathbb{R}}}
\newcommand{\scalar}[2]{\left<#1,#2\right>}          %   Dot product
\newcommand{\norm}[1]{\left\lVert #1 \right\rVert}   %   Norm

\DeclareMathOperator{\ArgminA}{Argmin}
\DeclareMathOperator*{\ArgminB}{Argmin}

%
%
%

\begin{document}
\maketitle
%
%
%
\begin{frame}[fragile]{Списки}
 Все списки задаются командами вида:
\begin{verbatim}
 \begin{name}
  \item Элемент списка 1
  \item Элемент списка 2
  ...
 \end{name}
\end{verbatim}
Списки бывают:
\begin{itemize}
 \item Ненумерованыне (\texttt{itemize});
 \item Нумерованные (\texttt{enumerate});
 \item С описанием (\texttt{description}).
\end{itemize}


У команды \verb:\item: есть необязательный аргумент, который, будучи указанным, заменяет собой номер (или пульку).

\end{frame}
%
%
%
\begin{frame}[fragile]{Библиография}
 Специальный тип списка --- список источников (библиография). Он задаётся окружением \texttt{thebibliography}, у которого есть один обязательный аргумент --- максимальное число элементов в списке.
Вместо \texttt{item} используется команда \texttt{bibitem} с одним обязательным аргументом --- ключом источника, необходимым для организации ссылок. Так, код
\begin{verbatim}
\begin{thebibliography}{99}
 \bibitem{Zorich} В.А. Зорич. Математический анализ. 
                   М.: Изд-во МЦНМО, 2007.
\end{thebibliography}
\end{verbatim}
позволит ссылаться в любом месте текста на источник с помощью команды \verb:\cite{Zorich}:.
\end{frame}


%
%
%
\begin{frame}{Многострочные формулы}
 Везде подразумевается, что пакеты семейства \texttt{ams} подключены. 
\begin{itemize}
 \item \texttt{gather} --- несколько идущих подряд формул.
 \item \texttt{multline} --- выкладки и / или преобразования, которые занимают несколько строчек.
 \item \texttt{align}, \texttt{aligned} --- системы (однотипных) уравнений.
 \item \texttt{pmatrix}, \texttt{vmatrix}, \texttt{bmatrix} --- набор матриц.  
 \item \texttt{cases} --- набор ситуаций вида <<если, то>>.
\end{itemize}
Каждое окружение надо использовать для решения задач, ему отвечающих.
\end{frame}
%
%
%
\begin{frame}[fragile]{Определения, теоремы и т.д.: пакет \texttt{amsthm}}
 
Сначала в преамбуле задаются новые команды, опредедяющие доступные окружения:

\verb:\theoremstyle{stylename} % задает текущий стиль:

\verb:\newtheorem{envName}{Открывающее слово}:

Каждая <<теорема>> имеет свой собственный счётчик и поддерживает ссылочный механизм. 

Стили можно задавать командой \verb:\newtheoremstyle:.

Использование в тексте:
\begin{verbatim}
 \begin{envName}
         Текст, текст, текст\ldots
 \end{envName}
\end{verbatim}
\end{frame}
%
%
%
\begin{frame}[fragile]{Новые команды}
 Если какая-то конструкция часто повторяется в тексте, то её лучше вынести в отдельную команду. Новые команды задаются в преамбуле:

\verb:\newcommand{\comName}{code}: --- команда без аргументов,

\verb:\newcommand{\comName}[nargs]{code}: --- команда с \texttt{nargs} аргументами.

После определения новыми командами можно пользоваться точно также, как и стандартными.
\end{frame}
%
%
%
\begin{frame}[fragile]
\begin{itemize}
\item Множество вещественных чисел:

\verb:\newcommand{\R}{\ensuremath{\mathbb{R}}}:

Результат: \R\  (\verb:\R:). Архаичный вариант, до сих пор встречающийся в литературе: $\mathrm{I\!R}$ (\verb:\mathrm{I\!R}:)
\item Норма:

\verb:\newcommand{\norm}[1]{\left\lVert #1 \right\rVert}: Скобки автоматически масштабируются:
$$
\norm{a + b},\ \norm{\int\limits_a^b f(x)\,dx},\ \norm{\dfrac{1}{42}}
$$
\item Скалярное произведение:

\verb:\newcommand{\scalar}[2]{\left<#1,#2\right>}:
 
Использование: \verb:\scalar{A}{B}:
\end{itemize}
\end{frame}
%
%
%
\begin{frame}[fragile]{Математические операторы}
Предположим, что мы хотим записать математическую команду <<множество минимизаторов>>. 
\begin{itemize}
\item В лоб: \verb:Argmin f(x): даёт $Argmin f(x)$.

\item Прямой шрифт: \verb:\mathrm{Argmin} f(x): даёт $\mathrm{Argmin} f(x)$.

\item Отбивка: \verb:\mathrm{Argmin}\,f(x): даёт $\mathrm{Argmin}\,f(x)$.

\end{itemize}
А если мы хотим записать $\ArgminB\limits_{x\in X}f(x)$?

\end{frame}
%
%
%
\begin{frame}[fragile]{\texttt{DeclareMathOperator}}
Решение: команды 
\begin{itemize}
\item \verb:\DeclareMathOperator{\Argmin}{Argmin}: --- для постановки индексов типа $\ArgminA\limits_{x\in X}$,

\item \verb:\DeclareMathOperator*{\Argmin}{Argmin}: --- позволяет использовать команду \verb:limits::
\begin{itemize}
\item \verb:\Argmin_{x\in X}: : $\ArgminB_{x\in X}$

\item \verb:\Argmin\limits_{x\in X}: : $\ArgminB\limits_{x\in X}$
\end{itemize}
\end{itemize}
\end{frame}


%
%
%
\begin{frame}[fragile]{Переопределение существующих команд}
 Ни \texttt{newcommand}, ни \texttt{DeclareMathOperator} не позволяют переопределять уже существующие команды. Для этого служит команда 
\texttt{renewcommand}. Примеры:
\begin{itemize}
 \item \verb:\renewcommand{\epsilon}{\varepsilon}: : замена $\epsilon$ на $\varepsilon$.
 \item \verb:\renewcommand{\phi}{\varphi}: : замена $\phi$ на $\varphi$.
 \item \verb:\renewcommand{\le}{\leqslant}: : замена $\le$ на $\leqslant$.
 \item \verb:\renewcommand{\ge}{\geqslant}: : замена $\ge$ на $\geqslant$.
 \item \verb:\renewcommand{\baselinestretch}{2}: : двойной межстрочный интервал во всём тексте.
\end{itemize}
\end{frame}
\begin{frame}[fragile]{Замена команд}
 Предположим, что мы хотим, чтобы вместо $\phi$ подставлялось $\varphi$, но не хотим терять доступ к исходному символу.

Что сделает следующий код?

\begin{verbatim}
\newcommand{\tmp}{\phi}
\renewcommand{\phi}{\varphi}
\renewcommand{\varphi}{\tmp}
\end{verbatim}


\end{frame}
\begin{frame}[fragile]{Низкоуровневая работа с макросами}
 Можно поступить так:
\begin{verbatim}
\let\temp\phi
\let\phi\varphi
\let\varphi\temp
\end{verbatim}

Или так:
\begin{verbatim}
\expandafter\mathchardef\expandafter\varphi\number
\expandafter\phi\expandafter\relax
\expandafter\mathchardef\expandafter\phi\number\varphi
\end{verbatim}
(Это одна команда.)

\end{frame}

%
%
%
\begin{frame}[fragile]{Размеры печатной области}
 По умолчанию у латеховских документов довольно приличные поля, и может возникнуть желание изменить их размер.

\textbf{Метод 1.} Задание всех параметров в преамбуле вручную:
\begin{verbatim}
\oddsidemargin=0cm
\evensidemargin=0cm
\topmargin=0cm
\textwidth=16.5cm
\headheight=0cm
\headsep=0cm
\topskip=0cm
\hoffset=0cm
\voffset=0.1cm
\textheight=25cm
\end{verbatim}
См. стр.166 в Львовском.
\end{frame}
%
%
%
\begin{frame}[fragile]
 \textbf{Метод 2.} Использовать пакет \texttt{geometry}:

\begin{verbatim}
 \usepackage[left=2cm,right=2cm,
    top=2cm,bottom=2cm,bindingoffset=0cm]{geometry}
\end{verbatim}

Модель страницы можно посмотреть в описании пакета (прилагается в материалах лекции).
\end{frame}
%
%
%
\begin{frame}[fragile]{Добавление графики}

В преамбулу добавить \verb:\usepackage[dvips]{graphicx}: для векторной графики и \verb:\usepackage[pdftex]{graphicx}: для растровой.

После этого команда

\verb:\includegraphics{fileNameWithoutExtension}:

вставит на своё место в документе графический файл из своего аргумента.

Полезные опции:

\verb:\includegraphics[keepaspectratio, width=0.8\textwidth]{file}:

\end{frame}

\begin{frame}[fragile]{Плавающие объекты}
Плавающий объект --- объект, расположение которого в документе остаётся на усмотрение транслятора. 

Для корректной работы в преамбуле надо подключить пакет \texttt{float}. Содержимое объекта помещается внутрь окружения \texttt{figure}:

\begin{verbatim}
\begin{figure}[пожелание]
   % Содержимое: картинка (или несколько картинок)
   \caption{Подпись}
\end{figure}
\end{verbatim}
Пожелание: одна буква из \texttt{t, b, p, h, T, B, P, H}.
\end{frame}


%
%
%
\begin{frame}{Домашнее задание}
\begin{itemize}
 \item Читать Львовского, стр. 98--123, 155-173, 185--193, 234--267.
 \item  Набрать страницу с номером, выпавшим Вам. Набор должен быть осуществлён в соответствии с Требованиям к Набору Отчётов. Номера формул можно ставить своими. Результат отправить на почту \url{month_october@mail.ru} для проверки.
\end{itemize}

\end{frame}

\end{document}

%
%
% 	Пример 01
%

%
%	С этой команды начинается каждый документ LaTeX: она задаёт общие настройки формата документа
%	Формат бумаги А4, 12й кегль (размер шрифта)
%
\documentclass[a4paper,12pt]{article}

%
%	Эти две команды подключают локализацию
%
\usepackage[utf8]{inputenc}			% Кодировка символов --- Юникод.
\usepackage[english, russian]{babel}		% Язык: шрифты, переносы и т.д.

%
%	Эти команды регулируют внешний вид документа
%

\usepackage{indentfirst}	% Абзацный отступ после начала глав и разделов
\usepackage{a4wide}		% Сужает поля

%
%	Эти команды подключают много полезного для работы с формулами
%

\usepackage{amsmath}
\usepackage{amsthm}
\usepackage{amssymb}


%
%	Для наглядности в этом тексте некоторые его части будут заключены в рамки.
%	Следующая команда это реализует.
%
\newcommand{\example}[1]{%
	\fbox{\parbox{0.9\textwidth}{#1}}
}

%
%	В одном из примеров будет использоваться скалярное произведение. Это команда задаёт его
%
\newcommand{\scalar}[2]{\left<#1,#2\right>}
%
%	Основное тело документа
%
\begin{document}
%
%	Варианты окружений <<со звёздочкой>> не оставляют после себя номеров
%
\section{Окружение \texttt{gather}}
Несколько подряд идущих \verb:$$::

\example{
$$
 F(x(\cdot),u(\cdot))\to\min,
$$
$$
\dot{x}(t) = f(x(t), u(t), t),
$$ $$
x(t_0) = x_0,
$$ $$
u(t)\in \mathcal{P}(t).
$$
}

При помощи \texttt{gather}:

\example{
\begin{gather}
F(x(\cdot),u(\cdot))\to\min,\\
\dot{x}(t) = f(x(t), u(t), t),\label{AA}\\
 x(t_0) = x_0,\notag\\
u(t)\in \mathcal{P}(t).
\end{gather}}

\eqref{AA}
\section{Окружение \texttt{multline}}
\begin{multline}
J(u_k) - J(v_k) =\\= 
\int\limits_{t_0}^T\scalar{u_k(\tau)}{u_k(\tau)}d\tau - \int\limits_{t_0}^T\scalar{\alpha_k\overline{u}(\tau) + (1-\alpha_k)u_k}{\alpha_k\overline{u}(\tau) + (1-\alpha_k)u_k}d\tau =\\=  
(2\alpha_k - \alpha_k^2)\int\limits_{t_0}^{T}\scalar{u_k(\tau)}{u_k(\tau)}d\tau - \biggl(2\alpha_k(1-\alpha_k)\int\limits_{t_0}^{T}\scalar{\overline{u}(\tau)}{u_k(\tau)}d\tau  \leqslant\\
\leqslant \left(|(2\alpha_k - \alpha_k^2)| R^2\biggr) + 2\alpha_k|(1-\alpha_k)|R^2 + \alpha_k^2R^2\right) \to 0,\ k\to\infty,
\end{multline}

\section{Системы уравнений}
Окружение \texttt{gather}:
\begin{gather*}
 (2- \alpha)x + 8y^2 = 8,\ (\text{из уравнения состояния})\\
3x + \dfrac{\alpha}{8}y = 2\ (\text{свойство консервативности системы}). 
\end{gather*}


Окружение \texttt{align}:
\begin{align}
(2- \alpha)x + 8y^2 &= 8 &(\text{из уравнения состояния})\\
3x + \dfrac{\alpha}{8}y &= 2 &(\text{свойство консервативности системы}) 
\end{align}

Команда \texttt{aligned}:
$$
(x- 4y^3 -9)(e^x + 2y -3) = 0\ \Longleftrightarrow \left[\begin{aligned}
                                    x- 4y^3 &= 9,\\
				     e^x + 2y &= 3.
                                    \end{aligned}\right.
$$

Окружением \texttt{eqnarray} пользоваться не рекомендуется.

\section{Ситуации <<или--или>>}
Окружение \texttt{aligned}:
$$
I(x,A) = \left\lbrace\begin{aligned}
          1, &\ x\in A,\\
	  0, &\ x\notin A.
         \end{aligned}\right.
$$
Окружение \texttt{cases}:
$$
I(x,A) = \begin{cases}
          1, & x\in A,\\
	  0, & x\notin A.
         \end{cases}
$$
\section{Матрицы}
Окружение \texttt{aligned}:
$$
\left[\begin{aligned}
          1\ & 2\ & 3\\
	  1\ & 2\ & 3
         \end{aligned}\right]
$$
Окружение \texttt{pmatrix}. \texttt{bmatrix}:
$$
 \begin{bmatrix}
          1 & 2 & 3\\
	  0 & 2 & 3
 \end{bmatrix},\ 
 \begin{pmatrix}
          1 & 2 & 3\\
	  0 & 2 & 3
 \end{pmatrix}
$$
\end{document} 

%
%
% 	Пример 02
%

%
%	С этой команды начинается каждый документ LaTeX: она задаёт общие настройки формата документа
%	Формат бумаги А4, 12й кегль (размер шрифта)
%
\documentclass[a4paper,12pt]{article}

%
%	Эти две команды подключают локализацию
%
\usepackage[utf8]{inputenc}			% Кодировка символов --- Юникод.
\usepackage[english, russian]{babel}		% Язык: шрифты, переносы и т.д.

%
%	Эти команды регулируют внешний вид документа
%

\usepackage{indentfirst}	% Абзацный отступ после начала глав и разделов
\usepackage{a4wide}		% Сужает поля

%
%	Эти команды подключают много полезного для работы с формулами
%

\usepackage{amsmath}
\usepackage{amsthm}
\usepackage{amssymb}

%
%	Тут мы задаём команды для теорем и определений, 
%	используя стандартные стили оформления
%
\theoremstyle{definition}
\newtheorem{defin}{Определение}
\theoremstyle{remark}
\newtheorem{rmk}{Замечание}
\theoremstyle{theorem}
\newtheorem{thm}{Теорема}
\newtheorem{lem}{Лемма}

%
%	Задаём собственный стиль оформления
%

\newtheoremstyle{SovietTheorem}% name of the style to be used
  {3pt}% measure of space to leave above the theorem. E.g.: 3pt
  {3pt}% measure of space to leave below the theorem. E.g.: 3pt
  {\itshape}% name of font to use in the body of the theorem
  {\parindent}% measure of space to indent
  {\bfseries}% name of head font
  {.}% punctuation between head and body
  {.5em}% space after theorem head; " " = normal interword space
  {}

\theoremstyle{SovietTheorem}
\newtheorem{sthm}{Теорема}

%
%	Основное тело документа
%
\begin{document}
%
%	Варианты окружений <<со звёздочкой>> не оставляют после себя номеров
%
\noindent Последняя строчка текста до определения.
\begin{defin}\label{separDefin}
Пространство называется сепарабельным, если в нём существует счётное всюду плотное множество.
\end{defin}
Первая строчка после определения.

\noindent Последняя строчка текста до замечания.
\begin{rmk}
Существуют пространства, не являющиеся сепарабельными. Иногда свойство из определения \ref{separDefin} добавляют к определению гильбертового пространства.
\end{rmk}
Первая строчка после замечания.
\begin{equation}\label{Taylor}
f(x) = f(x_0) + f'(x_0)(x-x_0) + \dfrac{1}{2}f''(x_0)(x-x_0)^2 + o(x^2).
\end{equation}
\noindent Последняя строчка текста до теоремы.
\begin{thm}
Любой, прочитавший текст определения \ref{separDefin}, знает, что такое сепарабельное пространство. Всякий, кто видел формулу \eqref{Taylor}, вспомнил ряд Тейлора.
\end{thm}
\begin{proof}
 Очевидно.
\end{proof}
Первая строчка после теоремы.

\noindent Последняя строчка текста до леммы.
\begin{lem}
Матричное уравнение $AX - BX = C$ имеет единственное решение для любой правой части $C$ тогда и только тогда, когда у матриц $A$ и $B$ нет общих собственных чисел.
\end{lem}
Первая строчка после леммы.

\noindent Последняя строчка текста до теоремы нового типа.
\begin{sthm}
Отечественный стиль оформления теорем, во-первых, ставит абзацный отступ перед словом <<Теорема>>, и, во-вторых, уменьшает объём пустого места до и после текста теоремы.
\end{sthm}
Первая строчка текста после теоремы нового типа.

\end{document} 
